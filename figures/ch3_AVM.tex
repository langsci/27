\documentclass[12pt]{article}
\usepackage[latin1]{inputenc}

\usepackage{xyling}

\usepackage{avm}
\usepackage{avm+}

\usepackage{combelow}

\usepackage{tree-dvips}
%\usepackage{qtree}


\begin{document}


\noindent{\textbf{(113) \hspace{1cm} Structure g\'en\'erale MRS}}
\vspace{3mm}

\noindent
\begin{avm}
[\tp{mrs}\\
\underline{hook} & [\underline{l}ocal-\underline{top} & $\rightarrow$ handle de l'EP ayant la port\'ee la plus large\\
			  \underline{ind}ex & $\rightarrow$ indice r\'ef\'erentiel \(type individu\(x\) ou \'ev\'enement\(e\)\)]\\
\underline{rel}ation\underline{s} & $\rightarrow$ une liste d'EP\\
\underline{h}andle-\underline{cons}traints & $\rightarrow$ un ensemble de contraintes de port\'ee]
\end{avm}


\bigskip



\noindent{\textbf{(115) \hspace{1cm} Repr\'esentation MRS du fragment en (116a)}}
\vspace{3mm}

\noindent
\begin{avm}
[source & [hook & [ind & @1]\\
           rels & @{A} <[\textit{venir-rel}\\
                                     ind @1\\
                                     arg1 @2],
                         [\textit{plusieurs-rel}\\
                                      ind @2],
                         [\textit{personne-rel}\\
                                       ind @2]>]\\
 abstract-cont & [hook & [ind @3]\\
               rels & @{B} <[\textit{venir-rel}\\
                            					ind @3]>]\\
 fragment & [rels & @{C} <[\textit{marie-rel}\\
                            				ind @4],
                           [\textit{hier-rel}\\
                                    event @5]>]\\
 target & [hook & [ind @5]\\
           rels & @{D} <[\textit{venir-rel}\\
                                  ind @5\\
                                  arg1 @4],
                         [\textit{marie-rel}\\
                                  ind @4],
                         [\textit{hier-rel}\\
                                  event @5]>]]
\end{avm}\\
\ \& \avmbox B est une distillation maximale de \avmbox A et \avmbox D\\ \& \avmbox D est une fusion de \avmbox B et \avmbox C


\bigskip



\noindent{\textbf{(116) \hspace{1cm} Repr\'esentation MRS du fragment (cas g\'en\'eral)}}
\vspace{3mm}

\noindent
\begin{avm}
[cont & [rels & @1]\\
 c-cont &  [rels & @2]\\
 fragment & [source &  [rels & @3]\\ 
             abstract-cont & [rels & @4]]\\ 
 dtrs & <[cont & [rels & @5]]>]
\end{avm}\\
\ \& \avmbox4 est une distillation maximale de \avmbox3 et \avmbox1\\ \& \avmbox1 est une fusion de \avmbox4, \avmbox5 et \avmbox2


\bigskip



\noindent{\textbf{(117) \hspace{1cm} Syntagme de type t\^ete-fragment dans Ginzburg \& Sag (2000)}}
\vspace{1mm}

\noindent
\begin{avm}
[\tp{head-fragment-ph} \\
category & [head & [\tp{verbal} \\ vform & finite]] \\
content & message \\
context & [sal-utt & \{[category @1 \\ content | index & @2]\}]] $\longrightarrow$ [category & @1 [head & nominal] \\ content & [index @2]]
\end{avm}

		
\bigskip


		
\noindent{\textbf{(122) \hspace{0.8cm} Syntagme de type t\^ete-fragment (nouvelle version)}
\vspace{3mm}

\noindent
\textit{head-fragment-ph} $\Rightarrow$
\begin{avm}
[category & [head & [cluster & <[head & @{H$_1$}],..., [head & @{H$_n$}]>]] \\
content & message\\
dtrs & <[\tp{cluster-ph}\\ head & [cluster & <[head & @{H$_1$}],..., [head & @{H$_n$}]>]]>]
\end{avm} 


\bigskip



\noindent{\textbf{(123) \hspace{1cm} Syntagme de type \textit{cluster}}
\vspace{3mm}

\noindent
\textit{cluster-ph} $\Rightarrow$ \textit{non-headed-ph} \& \\
\begin{avm}
[head & [\tp{head}\\
				 cluster & nelist\(synsem\) <@{1}, ..., @{n}>]\\
 n-hd-dtrs & <[synsem @{1}], ..., [synsem @{n}]>]
 
\end{avm}\\
%\& n $\geq$ 1 


\bigskip



\noindent{\textbf{(132) \hspace{1cm} Syntagme de type t\^ete-foncteur}}
\vspace{2mm}

\noindent
\textit{head-functor-ph} $\Rightarrow$ \textit{headed-ph} \& \\
\begin{avm}
[synsem & [cat & [head & @1\\
              mrkg & @2]]\\
 dtrs & <[synsem & [cat & [head & [select & @3]\\
                        	 mrkg & @2]]], @4>\\
 hd-dtr & @4 [synsem & @3 [cat & [head & @1]]]]
\end{avm}


\bigskip



\noindent{\textbf{(136) \hspace{1cm} Contrainte de localit\'e 1}}
\vspace{2mm}

\noindent 
\begin{avm}
[\tp{word}\\
 deps & <..., [cont [hook & @1 anchor]], ...>]
\end{avm}
$\Rightarrow$
\begin{avm}
[ss [cont [anchors & <..., @1, ...>]]]
\end{avm}


\bigskip



\noindent{\textbf{(137) \hspace{1cm} Contrainte de localit\'e 2}}
\vspace{2mm}

\noindent 
\begin{avm}
[\tp{cluster}\\
 dtrs & <..., [cont [hook & @1 anchor]], ...>]
\end{avm}
$\Rightarrow$
\begin{avm}
[ss [cont [anchors & <..., @1, ...>]]]
\end{avm}


\newpage



\noindent{\textbf{(139) \hspace{1cm} La construction RSV}}
\vspace{2mm}

\noindent
\textsc{vra}-\textit{ph} $\Rightarrow$ \textit{headed-ph} \& \\
\begin{avm} 
[cat & [head & @{1} [select & [cont &    [anchors &  <..., [ind & @{10}], ...> \\ rels & @{A} ]]]\\
									  val & @{2}\\
									  mrkg & @{9}]\\
						 cont & [hook & [ind & @{7}] \\ rels & <@{11}, @{12}> & $\oplus$ @{D}]\\
						 c-cont & [rels & <@{12}[\tp{sum-subpart-rel}\\
							 										subpart & @{7} event\\
							 										sum & @{6} event]>]\\
hd-dtr & @{4}[\tp{cluster-ph}\\ ss & @{3} [cat & [head & @{1}\\
																	val & @{2}\\
																	mrkg & unmarked]\\
													 cont & [hook & [ind & @{7}]\\
																				anchors & <..., [ind & @{8}], ...> \\ rels & @{C}]\\
													 fragment & [source [hook & [ind & @{6}] \\ rels & @{A}]\\
													  						abstract-cont  @{B}]\\
									 ]\\
									 ]\\
dtrs & <[cat & [head & [select & @3] \\ mrkg & @{9}]\\
	      cont & [hook & [ltop & @{5}]\\
								rels & < @{11}[\tp{sum-subpart-rel}\\
												label & @{5}\\
												subpart & @{8}\\
							 					sum & @{10}]>]], @{4}>]
													
\end{avm}	


\bigskip


   
\noindent{\textbf{(140) \hspace{1cm} La construction RSV avec un introducteur contenant une forme \textit{qu-}}
\vspace{2mm}

\noindent
\textsc{wh}-\textsc{vra}-\textit{ph} $\Rightarrow$ \textsc{vra}-\textit{ph} \& \\
\begin{avm}
[ss [cat [head [select [cont & [anchors & <..., [\textsc{ind} @1], ...>]]]]]\\
 dtrs <[rel \{@1\}], \textit{sign}>] 
\end{avm}


\bigskip



\noindent{\textbf{(143) \hspace{1cm} Entr\'ee lexicale pour la pr\'eposition \textit{dintre} en roumain}}
\vspace{2mm}

\noindent
\textsc{dintre}-\textit{word} $\Rightarrow$ \textit{word} \& \\
\begin{avm}
[cat & [head & [\tp{preposition}\\
								xarg & @{1}\\
								select & [cat & [head & [\tp{noun}\\
																 case & direct]]\\
										cont & [anchors & <..., [ind & @{3}], ...>]]]\\
				mrkg & dintre \\
				val & [comps & <@{2}>]\\
				arg-st & <@{1}[cat & [head & noun\\
														  val & [spr & <>]]\\
											 cont & [hook &[ind & @{3}]]], @{2}[cat & [head & noun\\
														  																	 val & [spr & <>]]\\
											 																		cont & [hook &[ind & @{4}]]\\
																													rel & \{@{4}\}]>]\\
cont & [hook & [ltop & @{5}]\\
				rels & <[\tp{partitioning-sum-subpart-rel}\\
                 label & @{5}\\
                 subpart & @{3}\\
                 sum & @{4}]>]]
\end{avm}


\bigskip



\noindent{\textbf{(144) \hspace{1cm} Entr\'ee lexicale pour la pr\'eposition \textit{printre} en roumain}}
\vspace{2mm}

\noindent 
\textsc{printre}-\textit{word} $\Rightarrow$ \textit{word} \& \\
\begin{avm}
[cat & [head & [\tp{preposition}\\
								xarg & @{1}\\
								select & [cont & [anchors & <..., [ind & @{3}], ...>]]]\\
				mrkg & printre \\
				val & [comps & <@{2}>]\\
				arg-st & <@{1}[cat & [head & noun\\
														  val & [spr & <>]]\\
											 cont & [hook &[ind & @{3}]]], @{2}[cat & [head & noun\\
														  																	 val & [spr & <>]]\\
											 																		cont & [hook &[ind & @{4}]]\\															rel & \{@{4}\}]>]\\							
cont & [hook & [ltop & @{5}]\\
				rels & <[\tp{sum-subpart-rel}\\
                 label & @{5}\\
                 subpart & @{3}\\
                 sum & @{4}]>]]
\end{avm}


\bigskip



\noindent{\textbf{(145) \hspace{1cm} Entr\'ee lexicale pour la pr\'eposition \textit{parmi} en fran\c{c}ais}}
\vspace{2mm}

\noindent 
\textsc{parmi}-\textit{word} $\Rightarrow$ \textit{word} \& \\
\begin{avm}
[cat & [head & [\tp{preposition}\\
								xarg & @{1}\\
								select & [cat & [head & noun]\\
								cont & [anchors & <..., [ind & @{3}], ...>]]]\\
								mrkg & parmi \\
				val & [comps & <@{2}>]\\
				arg-st & <@{1}[cat & [head & noun\\
														  val & [spr & <>]]\\
											 cont & [hook &[ind & @{3}]]], @{2}[cat & [head & noun\\
														  																	 val & [spr & <>]]\\
											 																		cont & [hook &[ind & @{4}]]\\
									rel & \{@{4}\}]>]\\								
cont & [hook & [ltop & @{5}]\\
				rels & <[\tp{sum-subpart-rel}\\
                 label & @{5}\\
                 subpart & @{3}\\
                 sum & @{4}]>]]
\end{avm}


\bigskip



\noindent{\textbf{(146) \hspace{1cm} La construction RSV avec un introducteur contenant \textit{dont} en fran\c{c}ais}}
\vspace{2mm}

\noindent 
\textsc{dont}-\textsc{vra}-\textit{ph} $\Rightarrow$ \textsc{vra}-\textit{ph} \& \\
\begin{avm}
[ss [cat [mrkg & dont]]] 
\end{avm}


\bigskip



\noindent{\textbf{(147) \hspace{1cm} Entr\'ee lexicale pour \textit{dont} dans les RSV en fran\c{c}ais}}
\vspace{2mm}

\noindent 
\textsc{dont}-\textit{word} $\Rightarrow$ \textit{word} \& \\
\begin{avm}
[cat & [head & [select & [cat & [head & [select & [cont & [anchors & <..., [ind & @{4}], ...>]]]]\\
													cont & [anchors & <..., [ind & @{3}], ...>]]]\\
				val & <>\\
				mrkg & dont]\\
cont & [hook & [ltop & @{5}]\\
				rels & <[\tp{sum-subpart-rel}\\
                 label & @{5}\\
                 subpart & @{3}\\
                 sum & @{4}]>]]
\end{avm}


\bigskip



\noindent{\textbf{(148) \hspace{1cm} Entr\'ee lexicale pour le compl\'ementeur \textit{dont} dans les relatives verbales ordinaires}}
\vspace{3mm}

\noindent 
\begin{avm}
[head & [\tp{complementizer}\\
				 vform & @{1}]\\
mrkg & dont\\
comps & <S & [vform & @{1}tensed\\
							subj & <>\\
							slash & \{@{2}PP[\textit{de}]\}\\
							mrkg & unmarked]>\\
bind & \{@{2}\}\\ 
slash & \{\}]
\end{avm} 






\end{document}