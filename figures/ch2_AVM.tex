\documentclass[12pt]{article}
\usepackage[latin1]{inputenc}
\usepackage{libertine}

\usepackage{fullpage}

\usepackage{xyling}

\usepackage{avm}
\usepackage{avm+}

\usepackage{combelow}

\usepackage{tree-dvips}
%\usepackage{qtree}
\usepackage{soul} %for strikethrough font

\begin{document}

\bigskip

\noindent{\textbf{(245) \hspace{1cm} R\`egles du gapping dans Culicover \& Jackendoff (2005)}
\vspace{3mm}

\noindent
Syntaxe:
\begin{avm}
[XP$_{i}$$^{ORPHAN1}$ YP$_{j}$$^{ORPHAN2}$]$^{IL}$
\end{avm}\\	
Structure conceptuelle (CS):	
\begin{avm}
[$\mathcal{F}$([X$_{i}$\\c-focus], [Y$_{j}$\\c-focus])]
\end{avm}	


\bigskip



\noindent{\textbf{(246) \hspace{1cm} La s\'emantique du gapping dans Culicover \& Jackendoff (2005)}	
\vspace{3mm}

\noindent
\begin{avm}
[[speak ([robin\\c-focus], [french\\c-focus])] and[$\mathcal{F}$([leslie\\c-focus], [german\\c-focus])]]
\end{avm}	


\bigskip



\noindent{\textbf{(247) \hspace{1cm} Principe de conservation des arguments}
\vspace{3mm}

\noindent
\textit{word} $\Rightarrow$
\begin{avm}
[valence  & [subj & @1\\
						spr & @2\\
						comps & @3]\\
 arg-st & @1 $\oplus$ @2 $\oplus$ @3 $\bigcirc$ list\(non-canonical\)] 
\end{avm}	


\bigskip



\noindent{\textbf{(249) \hspace{1cm} Repr\'esentation simplifi\'ee de la phrase}
\vspace{3mm}

\noindent
\textit{clause} $\Rightarrow$ 
\begin{avm}
[cat & [val & [subj & <>\\
               comps & <>\\
               spr <>]]\\
cont & message]
\end{avm}


\bigskip



\noindent{\textbf{(250) \hspace{1cm} Entr\'ee lexicale d'une conjonction}}
\vspace{1mm}

\noindent
\textit{conj-word} $\Rightarrow$
\begin{avm}
[category & [head&@{1}\\
						marking&@{2}\\
						valence & [subj&@{3}\\
											 spr&@{4}\\
											 comps & <[head&@{1}\\
											 					marking&@{2}\\
											 					subj&@{3}\\
											 					spr&@{4}\\
											 					comps&@{5}\\
											 					conj&nil]> & $\oplus$ & @{5}]\\
						conj&$\neg$nil]]
						\end{avm}


\bigskip



\noindent{\textbf{(251) \hspace{1cm} R\`egle g\'en\'erale de la coordination}}
\vspace{3mm}

\noindent
\textit{coord-phrase} $\Rightarrow$ \textit{non-headed-ph} \&
\begin{avm}
[dtrs & <$sign$, $sign$> \, $\oplus$ list($sign$)]
\end{avm}
\& \\
\begin{avm}
[synsem & [conj & nil]\\
 dtrs & list([conj & nil]) \, $\oplus$  <[conj @1 & $\neg$nil],..., [conj & @1]   >]
 \end{avm}


\bigskip



\noindent{\textbf{(252) \hspace{1cm} Syntagme de type \textit{simplex-coord-ph}}}
\vspace{3mm}

\noindent
\textit{simplex-coord-phrase} $\Rightarrow$ \textit{coord-ph} \& \\
\begin{avm}
[dtrs & <[conj & nil]> \, $\oplus$ list($sign$) \, $\oplus$ <[conj & $\neg$nil] >]
 \end{avm}

\bigskip

		
\noindent{\textbf{(253) \hspace{1cm} Syntagme de type \textit{omnisyndetic-coord-ph}}}
\vspace{3mm}

\noindent
\textit{omnisyndetic-coord-phrase} $\Rightarrow$ \textit{coord-ph} \& \\
\begin{avm}
[dtrs & <[conj & $\neg$nil] \, $\oplus$ & list($sign$) >]
 \end{avm}


\bigskip



\noindent{\textbf{(254) \hspace{1cm} Syntagme de type \textit{asyndetic-coord-ph}}}
\vspace{3mm}

\noindent
\textit{asyndetic-coord-phrase} $\Rightarrow$ \textit{coord-ph} \& \\
\begin{avm}
[dtrs & list($sign$) \, $\oplus$ <[conj & nil]> ]
 \end{avm}


\bigskip



\noindent{\textbf{(258) \hspace{1cm} Contraintes de parall\'elisme dans les constructions coordonn\'ees}}
\vspace{3mm}

\noindent
\textit{coord-phrase} $\Rightarrow$
\begin{avm}
[synsem & [head / @H\\
						 valence @V\\
						 slash @S]\\
 dtrs & <[head / @H\\
					valence @V\\
					slash @S],...,
					[head / @H\\
					valence @V\\
					slash @S]>]
\end{avm}	


\bigskip



\noindent{\textbf{(262) \hspace{1cm} Syntagme de type \textit{cluster}}}
\vspace{3mm}

\noindent
\textit{cluster-ph} $\Rightarrow$ \textit{non-headed-ph} \& \\
\begin{avm}
[head & [\tp{head}\\
				 cluster & nelist\(synsem\) <@{1}, ..., @{n}>]\\
subj & < >\\
spr & < > \\
comps & < >\\
slash & $\Sigma$$_1$ $\cup$ ... $\cup$ $\Sigma$$_n$\\
n-hd-dtrs & <[synsem @{1} [slash & $\Sigma$$_1$]], ..., [synsem @{n} [slash & $\Sigma$$_n$]]>]
 \end{avm}


\bigskip



\noindent{\textbf{(268) \hspace{1cm} Le syntagme de type fragment dans Ginzburg \& Sag (2000)}}
\vspace{3mm}

\noindent
\begin{avm}
[\tp{head-fragment-ph} \\
category | head & [\tp{verbal} \\ vform & finite] \\
content & message \\
context | sal-utt & \{[category @1 \\ content | index & @2]\}] $\longrightarrow$ [category @1 [head & nominal] \\ content | index @2]
\end{avm}


\bigskip 



\noindent{\textbf{(271) \hspace{1cm} Contrainte syntaxique du \textit{head-fragment-ph}}
\vspace{3mm}

\noindent
\textit{head-fragment-ph} $\Rightarrow$
\begin{avm}
[context | sal-utt \{[head @{H$_1$} \\ major + ],..., [head @{H$_n$} \\ major + ]\} \\
category | head | cluster <[head & @{H$_1$}],..., [head & @{H$_n$}]>]
\end{avm}  


\bigskip



\noindent{\textbf{(275) \hspace{1cm} Contrainte s\'emantique du \textit{head-fragment-ph}}
\vspace{3mm}

\noindent 
\textit{head-fragment-ph} $\Rightarrow$
\begin{avm}[context [source & message @{M} \\ sal-utt & \{[content & @{C$_1$}],..., [content & @{C$_n$}]\}]\\
category | head [\tp{head}\\ cluster & <[content & @{C$_1$'}],..., [content & @{C$_n$'}]>]\\
content \textit{R}$_{sem}$(@{M}, <@{C$_1$}, @{C$_1$'}>,..., <@{C$_n$}, @{C$_n$'}>)]
\end{avm} 


\bigskip



\noindent{\textbf{(276) \hspace{1cm} La construction \`a gapping}
\vspace{3mm}

\noindent
\textit{gapping-ph} $\Rightarrow$ \textit{coord-ph} \& \\
\begin{avm} 
[head @{H} \textit{verbal}\\
context | background \{..., \textit{sym-discourse-rel}(@{M$_1$},..., @{M$_j$}, @{M$_{j+1}$},..., @{M$_n$}), ...\}\\
dtrs <[head @{H} [\tp{verbal}\\
									  cluster & elist]\\
content @{M$_1$}],..., [head @{H} [\tp{verbal}\\
									  cluster & elist]\\
content @{M$_j$}]> \, $\oplus$ \\

<[head [cluster & <@1,..., @n >]\\ source @{M$_j$}\\ content @{M$_{j+1}$}],..., [head [cluster & <@{1'},..., @{n'} >]\\ source @{M$_j$}\\ content @{M$_{n}$}]>]  
\end{avm}


\bigskip



\noindent{\textbf{(322) \hspace{1cm} Syntagme de type \textit{cluster}}}
\vspace{3mm}

\noindent
\textit{cluster-ph} $\Rightarrow$ \textit{non-headed-ph} \& \\
\begin{avm}
[head & [\tp{head}\\
				 cluster & nelist\(synsem\) <@{1}, ..., @{n}>]\\
subj & < >\\
spr & < > \\
comps & < >\\
slash & $\Sigma$$_1$ $\cup$ ... $\cup$ $\Sigma$$_n$\\
n-hd-dtrs & <[synsem @{1} [slash & $\Sigma$$_1$]], ..., [synsem @{n} [slash & $\Sigma$$_n$]]>]
 \end{avm}


\bigskip



\noindent{\textbf{(323) \hspace{1cm} R\`egle lexicale pour la compl\'ementation alternative des pr\'edicats}}
\vspace{3mm}

\noindent
\begin{avm}
[\tp{cluster-coord-lexical-rule}\\
input & [\tp{word}\\
				comps @{L$_1$} + & @{L$_2$} nelist <[cat & @{1}], ..., [cat & @{n}]>]\\
output & [\tp{word}\\
				comps @{L$_1$} + & <[coord + \\
													cluster <[cat & @{1}], ..., [cat & @{n}]>]>]]\end{avm}\\
\begin{avm}												
\& @{L$_2$} $\neq$ <[coord +\\ cluster & nelist\(synsem\)]>		
\end{avm}


\bigskip



\noindent{\textbf{(327) \hspace{1cm} Entr\'ees lexicales du verbe \textit{a da} 'donner'}}
\vspace{3mm}

\noindent
\textit{a da}$_1$: 
\begin{avm}
[comps <NP$_{acc}$> &  $\oplus$ <NP$_{dat}$>]
\end{avm}\\
\textit{a da}$_2$: 
\begin{avm}
[comps <[coord +\\ cluster <NP$_{acc}$> &  $\oplus$ <NP$_{dat}$>]>]
\end{avm}


\bigskip



\noindent{\textbf{(329) \hspace{1cm} Entr\'ees lexicales du verbe \textit{a scrie} '\'ecrire'}}
\vspace{3mm}

\noindent
\textit{a scrie}$_1$: 
\begin{avm}
[comps <NP$_{acc}$> &  $\oplus$ <(NP$_{dat}$)> ]
\end{avm}\\
\textit{a scrie}$_2$: 
\begin{avm}
[comps <[coord +\\ cluster <NP$_{acc}$> &  $\oplus$ <(NP$_{dat}$)>]>]
\end{avm}



						  				
										  				
\end{document}