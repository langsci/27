%%%%%%%%%%%%%%%%%%%%%%%%%%%%%%%%%%%%%%%%%%%%%%%%%%%%
%%%                                              %%%
%%%     Language Science Press Master File       %%%
%%%         follow the instructions below        %%%
%%%                                              %%%
%%%%%%%%%%%%%%%%%%%%%%%%%%%%%%%%%%%%%%%%%%%%%%%%%%%%
% Everything following a % is ignored% Some lines start with %. Remove the % to include them

\documentclass[output=book,
 nonflat,
 modfonts,
 nobabel
                 ]{langsci/langscibook}
  
   
%%%%%%%%%%%%%%%%%%%%%%%%%%%%%%%%%%%%%%%%%%%%%%%%%%%%
%%%                                              %%%
%%%          additional packages                 %%%
%%%                                              %%%
%%%%%%%%%%%%%%%%%%%%%%%%%%%%%%%%%%%%%%%%%%%%%%%%%%%%

% put all additional commands you need in the 
% following files. {I}f you do not know what this might 
% mean, you can safely ignore this section

\usepackage[ngerman]{babel}
\usepackage{qtree}
\usepackage{amsmath}
\usepackage{pst-jtree}
\usepackage{array} 
\usepackage{mdsymbol}
\usepackage{diagbox}
\usepackage{pst-3d}
\usepackage{graphicx}
\usepackage{hyperref}
\usepackage{arydshln}
\usepackage{tabularx}
\newcolumntype{x}[1]{!{\centering\arraybackslash\vrule width #1}}
\usepackage[demo]{graphicx}
\usepackage{booktabs}
\usepackage{setspace}\usepackage{threeparttable}
\usepackage{multirow}
\usepackage{makecell}


\title{Grammaire des constructions elliptiques}  %look no further, you can change those things right here.
\subtitle{Une étude comparative des phrases sans verbe en roumain et en français} 
\BackBody{Dans cet ouvrage, j’ai développé un fragment de grammaire qui rend compte
des propriétés majeures de deux constructions elliptiques (auxquelles
«~manque~» la tête verbale), appelées respectivement gapping
(1a) et relatives sans verbe (1b).
Les deux constructions ont été choisies pour montrer qu’une analyse
uniforme (sans reconstruction syntaxique) est disponible dans les deux
types de relations syntaxiques : coordination et subordination.

\ea
\ea
Jean aime les pommes [\textbf{et} Marie les bananes].  
\ex
Plusieurs personnes sont venues cette semaine, [\textbf{dont} Marie
(hier)].  
\z
\z

La proposition centrale défendue ici est que les phrases elliptiques dans
les coordinations à gapping et les subordonnées relatives sans verbe ne
peuvent pas être alignées sur le fonctionnement d’une phrase verbale
ordinaire. Leurs propriétés syntaxiques et sémantiques montrent qu’elles ne
sont pas dérivées à partir d’une phrase complète. Une analyse en termes de
reconstruction syntaxique est donc inadéquate. Par conséquent, la phrase
elliptique dans les deux types de structures mentionnés comporte un mode
d’organisation syntaxique spécifique et doit avoir un statut indépendant
dans la grammaire, à savoir le statut d’une \textit{phrase fragmentaire},
c.-à-d. une unité syntaxique qui a un contenu de type \textit{message}, mais dont la
syntaxe est incomplète. Cet ouvrage apporte de nouveaux arguments en faveur
d’une reconstruction plutôt sémantique avec des contraintes de
parallélisme, cf. \citet{GinzburgEtAl2000} et \citet{CulicoverEtAl2005}.}
%\dedication{Change dedication in localmetadata.tex}
\proofreader{Margot Colinet, Frédéric Laurens,
Sebastian Nordhoff, Florence Orluc, Delphine Tribout, Esma Tanis, Grégoire Winterstein}
\typesetter{Gabriela Bîlbîie, Sebastian Nordhoff, Felix Kopecky, Iana Stefanova}
\author{Gabriela Bîlbîie} 
\renewcommand{\lsISBNdigital}{978-3-944675-52-7}
\renewcommand{\lsISBNhardcover}{978-3-946234-82-1}
\renewcommand{\lsISBNsoftcover}{978-3-944675-67-1} 
\renewcommand{\lsSeries}{eotms} % use lowercase acronym, e.g. sidl, eotms, tgdi
\renewcommand{\lsSeriesNumber}{2} %will be assigned when the book enters the proofreading stage
\renewcommand{\lsURL}{http://langsci-press.org/catalog/book/27} % contact the coordinator for the right number
\newcommand{\lsID}{27}
\BookDOI{10.5281/zenodo.573753}

 
% add all extra packages you need to load to this file  
\usepackage{tabularx} 

%%%%%%%%%%%%%%%%%%%%%%%%%%%%%%%%%%%%%%%%%%%%%%%%%%%%
%%%                                              %%%
%%%           Examples                           %%%
%%%                                              %%%
%%%%%%%%%%%%%%%%%%%%%%%%%%%%%%%%%%%%%%%%%%%%%%%%%%%% 
%% to add additional information to the right of examples, uncomment the following line
% \usepackage{jambox}
%% if you want the source line of examples to be in italics, uncomment the following line
% \renewcommand{\exfont}{\itshape}
\usepackage{etoolbox}
\usepackage{langsci/styles/langsci-optional}

\usepackage{langsci-lgr}
\usepackage{todonotes}
 
\usepackage[linguistics]{forest}
\usepackage{tikz}
\usetikzlibrary{fit}
\usetikzlibrary{arrows.meta}


\usepackage{multicol}

\usepackage[normalem]{ulem}
\usepackage{soul}
\usepackage{avm}
\usepackage{tkz-graph} % more complicated graphs (aka trees)
\usepackage{langsci/styles/langsci-gb4e}
%% hyphenation points for line breaks
%% Normally, automatic hyphenation in LaTeX is very good
%% If a word is mis-hyphenated, add it to this file
%%
%% add information to TeX file before \begin{document} with:
%% %% hyphenation points for line breaks
%% Normally, automatic hyphenation in LaTeX is very good
%% If a word is mis-hyphenated, add it to this file
%%
%% add information to TeX file before \begin{document} with:
%% %% hyphenation points for line breaks
%% Normally, automatic hyphenation in LaTeX is very good
%% If a word is mis-hyphenated, add it to this file
%%
%% add information to TeX file before \begin{document} with:
%% \include{localhyphenation}
\hyphenation{
ac-cep-ta-ble
a-na-lyse
ap-pa-rais-sent %check the hyphenation
au-xi-liaire
clus-ter
clus-ters
com-pa-ra-tive
com-pa-ra-tives
con-si-dé-ré
cons-truc-tions
cons-truc-tion
con-texte
con-textes
con-tras-tif
con-ven-tion-nelles
deu-xième %check the hyphenation
E-di-tu-ra
en-tre-tien-nent
en-vi-sa-gé
é-ven-tua-li-té
e-xem-ple
e-xem-ples
Fien-go
Gar-dent
Han-ka-mer
in-fi-ni-tif
in-ter-pré-ta-tion
in-ter-ro-ga-ti-ves
Ka-ze-nin
Ken-ne-dy
ma-té-riel
par-ti-cu-lier
per-met-tant
prin-tre
pseu-do-cons-ti-tuants
pseu-do-gap-ping
quel-ques
re-cons-truc-tion
ren-tre-ra
Schlan-gen
sous-ca-té-go-ri-sa-tion
su-bor-don-nants
suf-fi-sam-ment
sui-vant
syn-ta-xique
moin-dre
}
\hyphenation{
ac-cep-ta-ble
a-na-lyse
ap-pa-rais-sent %check the hyphenation
au-xi-liaire
clus-ter
clus-ters
com-pa-ra-tive
com-pa-ra-tives
con-si-dé-ré
cons-truc-tions
cons-truc-tion
con-texte
con-textes
con-tras-tif
con-ven-tion-nelles
deu-xième %check the hyphenation
E-di-tu-ra
en-tre-tien-nent
en-vi-sa-gé
é-ven-tua-li-té
e-xem-ple
e-xem-ples
Fien-go
Gar-dent
Han-ka-mer
in-fi-ni-tif
in-ter-pré-ta-tion
in-ter-ro-ga-ti-ves
Ka-ze-nin
Ken-ne-dy
ma-té-riel
par-ti-cu-lier
per-met-tant
prin-tre
pseu-do-cons-ti-tuants
pseu-do-gap-ping
quel-ques
re-cons-truc-tion
ren-tre-ra
Schlan-gen
sous-ca-té-go-ri-sa-tion
su-bor-don-nants
suf-fi-sam-ment
sui-vant
syn-ta-xique
moin-dre
}
\hyphenation{
ac-cep-ta-ble
a-na-lyse
ap-pa-rais-sent %check the hyphenation
au-xi-liaire
clus-ter
clus-ters
com-pa-ra-tive
com-pa-ra-tives
con-si-dé-ré
cons-truc-tions
cons-truc-tion
con-texte
con-textes
con-tras-tif
con-ven-tion-nelles
deu-xième %check the hyphenation
E-di-tu-ra
en-tre-tien-nent
en-vi-sa-gé
é-ven-tua-li-té
e-xem-ple
e-xem-ples
Fien-go
Gar-dent
Han-ka-mer
in-fi-ni-tif
in-ter-pré-ta-tion
in-ter-ro-ga-ti-ves
Ka-ze-nin
Ken-ne-dy
ma-té-riel
par-ti-cu-lier
per-met-tant
prin-tre
pseu-do-cons-ti-tuants
pseu-do-gap-ping
quel-ques
re-cons-truc-tion
ren-tre-ra
Schlan-gen
sous-ca-té-go-ri-sa-tion
su-bor-don-nants
suf-fi-sam-ment
sui-vant
syn-ta-xique
moin-dre
}
\bibliography{localbibliography} 

%%%%%%%%%%%%%%%%%%%%%%%%%%%%%%%%%%%%%%%%%%%%%%%%%%%%
%%%                                              %%%
%%%             Frontmatter                      %%%
%%%                                              %%%
%%%%%%%%%%%%%%%%%%%%%%%%%%%%%%%%%%%%%%%%%%%%%%%%%%%% 
\begin{document}     
%add all your local new commands to this file

\newcommand{\smiley}{:)}

\renewbibmacro*{index:name}[5]{%
  \usebibmacro{index:entry}{#1}
    {\iffieldundef{usera}{}{\thefield{usera}\actualoperator}\mkbibindexname{#2}{#3}{#4}{#5}}}

% \newcommand{\noop}[1]{}

\newcommand{\textstyleapplestylespan}[1]{#1}

\setcounter{tocdepth}{3}
 
\defbibheading{bibliographie}{\chapter{Bibliographie}} 
\renewcommand{\chapref}[1]{Chapitre~\ref{#1}}
\renewcommand{\tabref}[1]{Tableau~\ref{#1}}
\renewcommand{\lsLanguageIndexTitle}{Index des langues}	% 
\renewcommand{\lsSubjectIndexTitle}{Index terminologique}
\renewcommand{\lsNameIndexTitle}{Index des auteurs}
\addto\captionsfrench{\def\tablename{Tableau}}
\renewcommand{\tablename}{Tableau}
\renewcommand{\sectref}[1]{Section~\ref{#1}}
\renewcommand{\chapref}[1]{Chapitre~\ref{#1}}
\renewcommand{\fnexfont}{\upshape}


% \newcommand{\rephrase}[2]{{\color{yellow!30!black}#2}\todo{replaced `#1'}}
\newcommand{\ulg}[2]{%#1: stuff to underline, no extra length
 \ulp{#1}{\hspace*{#2mm}}
 }

 \newcommand{\cad}{\mbox{c.-à-d.}}
  


\maketitle                
\frontmatter
% %% uncomment if you have preface and/or acknowledgements

\currentpdfbookmark{Contents}{name} % adds a PDF bookmark
\tableofcontents
% \addchap{Preface}
% \begin{refsection}
% 
% %content goes here
% 
% \printbibliography[heading=subbibliography]
% \end{refsection}
% \addchap{Remerciements}

Plusieurs personnes ont contribué, de près ou de loin, à l’aboutissement de cet ouvrage et je tiens à les en remercier chaleureusement. 

D’abord, je voudrais remercier \textit{Language Science Press} d’avoir accepté la publication de ce livre. Je tiens à remercier particulièrement Stefan Müller et Berthold Chrysmann pour l’intérêt qu’ils ont porté à mon travail et la patience avec laquelle ils ont attendu la version finale du manuscrit. Je remercie aussi les deux évaluateurs anonymes du manuscrit ; leurs commentaires très fins et leurs suggestions ont nettement amélioré la qualité de cet ouvrage. Un grand merci à Sebastian Nordhoff, qui m’a accompagnée tout au long du processus de mise en forme, afin de remplir les critères techniques de publication de \textit{Language Science Press}.

Cet ouvrage reprend une bonne partie de ma thèse de doctorat, soutenue à l’Université Paris Diderot – Paris 7 en 2011. Je voudrais exprimer toute ma reconnaissance à ma directrice de thèse, Anne Abeillé, qui a dirigé mes recherches depuis le master et m’a permis de réaliser mes travaux de recherche dans les meilleures conditions possibles. Ses intuitions très fines, sa souplesse et son ouverture d’esprit ont contribué de manière significative à mon parcours scientifique. 

J’ai eu la chance de me former au milieu d’une équipe exceptionnelle de chercheurs et d’enseignants-chercheurs du Laboratoire de Linguistique Formelle et de l’Université Paris Diderot – Paris 7. Comme doctorante et jeune chercheur, j’ai énormément profité de leurs savoirs et expérience. Plusieurs ont eu un apport essentiel et je tiens à les remercier en particulier. Je suis très reconnaissante à François Mouret, qui m’a aidée à comprendre plusieurs aspects liés à la coordination et à l’ellipse. Son suivi, le sérieux et la précision de ses commentaires m’ont toujours aidée à avancer dans mon travail. Je remercie aussi Olivier Bonami pour son aide précieuse sur la partie formelle de cet ouvrage ; ses commentaires m’ont aidée à résoudre certains problèmes relevant de la méthodologie, la description ou encore la formalisation de faits linguistiques. J’ai eu la chance de côtoyer Jean-Marie Marandin et Danièle Godard ; les discussions avec eux ont répondu à mes questions délicates et m’ont ouvert de nouveaux horizons scientifiques. Je remercie aussi Philip Miller d’avoir lu certaines sections de cet ouvrage ; ses suggestions très pointues m’ont aidée dans la rédaction du chapitre général dédié aux phrases elliptiques.  

Je remercie les personnes qui ont accepté d’être membres de mon jury de thèse. Je suis très reconnaissante à Carmen Dobrovie-Sorin pour toute son aide apportée à divers titres depuis mon arrivée à l’Université Paris 7 comme étudiante Erasmus. Je remercie Jonathan Ginzburg d’avoir soulevé des questions profondes liées à l’analyse formelle de l’ellipse. Je tiens à exprimer ma reconnaissance à Emil Ionescu, un des premiers professeurs m’ayant initiée à la linguistique à l’Université de Bucarest. Ma gratitude s’adresse aussi à Jason Merchant, qui, par l’intermédiaire de ses présentations et articles, m’a aidée à approfondir le champ d’étude de l’ellipse ; j’apprécie beaucoup son honnêteté intellectuelle et son ouverture d’esprit. Enfin, je remercie Marleen Van Peteghem d’avoir accepté de lire ma thèse malgré des conditions difficiles. 

Un grand merci à mes chers collègues et amis Frédéric Laurens et Grégoire Winterstein, qui ont eu un apport non négligeable aux résultats de cette thèse, par leurs collaborations et leurs discussions stimulantes. Merci à Margot Colinet pour sa disponibilité même à des heures indues, ma référence en matière de français. Merci à mes collègues et amis à Paris 7, et notamment à Israel de la Fuente, Anna Gazdik, Fabiola Henri, Jana Strnadova et Delphine Tribout. Merci à Clément Plancq, qui m’a soutenue dans mes problèmes informatiques.

Je clos enfin ces remerciements en dédiant cet ouvrage à ma famille et aux quelques amis très proches (sans oublier Daniel Lavalette), qui m’ont soutenue tout au long de ces années de travail. Et, plus que quiconque, je remercie Răzvan, mon soutien sans faille, d’avoir sublimé les 2500 km entre Bucarest et Paris pour venir à mes côtés.
% \addchap{Liste des abréviations utilisées dans les gloses des exemples}
 
\begin{multicols}{2}
\begin{tabbing}
distr \hspace{1em} \= marquage différentiel de l’objet direct\kill
1 \> première personne\\
2 \> deuxième personne\\
3 \> troisième personne\\
\textsc{acc} \> accusatif\\
\textsc{adj} \> adjectif\\
\textsc{adv} \> adverbe\\
\textsc{art} \> article\\
\textsc{aux} \> auxiliaire\\
\textsc{comp} \> complémenteur\\
\textsc{cond} \> conditionnel\\
\textsc{conj} \> conjonction\\
\textsc{dat} \> datif\\
\textsc{decl} \> déclarative\\
\textsc{def} \> défini\\
\textsc{dem} \> démonstratif\\
\textsc{det} \> déterminant\\
\textsc{distr} \> distributif\\
\textsc{dom} \> marquage différentiel de \\
	\> l’objet direct\\
\textsc{dur} \> aspect duratif\\
\textsc{excl} \> exclamative\\
\textsc{f} \> féminin\\
\textsc{foc} \> focus\\
\textsc{fut} \> futur\\
\textsc{gen} \> génitif\\
\textsc{imp} \> impératif/impérative\\
\textsc{ind} \> indicatif\\
\textsc{indf} \> indéfini\\
\textsc{inf} \> infinitif\\
\textsc{inter} \> interrogative\\
\textsc{intr} \> intransitif\\
\textsc{ipfv} \> imparfait\\
\textsc{m} \> masculin\\
\textsc{n} \> neutre\\
\textsc{neg} \> négation, négatif\\
\textsc{nom} \> nominatif\\
\textsc{obj} \> objet\\
\textsc{obl} \> oblique\\
\textsc{pass} \> passif\\
\textsc{perf} \> aspect perfectif\\
\textsc{pl} \> pluriel\\
\textsc{pred} \> prédicatif\\
\textsc{prf} \> parfait\\
\textsc{prs} \> présent\\
\textsc{pst} \> passé\\
\textsc{ptcp} \> participe\\
\textsc{q} \> marqueur interrogatif\\
\textsc{refl} \> réfléchi\\
\textsc{rel} \> relatif\\
\textsc{sbj} \> sujet\\
\textsc{sbjv} \> subjonctif\\
\textsc{sg} \> singulier\\
\textsc{sup} \> supin\\
\textsc{top} \> topique\\
\textsc{tr} \> transitif\\
\textsc{voc} \> vocatif\\ 
\end{tabbing}
\end{multicols} 
\mainmatter         
  

%%%%%%%%%%%%%%%%%%%%%%%%%%%%%%%%%%%%%%%%%%%%%%%%%%%%
%%%                                              %%%
%%%             Chapters                         %%%
%%%                                              %%%
%%%%%%%%%%%%%%%%%%%%%%%%%%%%%%%%%%%%%%%%%%%%%%%%%%%%
 
\include{chapters/01}  %add a percentage sign in front of the line to exclude this chapter from book
\include{chapters/02}
\include{chapters/03}
\include{chapters/04}
\include{chapters/05}
\include{chapters/06}


% % copy the lines above and adapt as necessary

%%%%%%%%%%%%%%%%%%%%%%%%%%%%%%%%%%%%%%%%%%%%%%%%%%%%
%%%                                              %%%
%%%             Backmatter                       %%%
%%%                                              %%%
%%%%%%%%%%%%%%%%%%%%%%%%%%%%%%%%%%%%%%%%%%%%%%%%%%%%

\is{some term| see {some other term}}
\il{some language| see {some other language}}
\issa{some term with pages}{some other term also of interest}
\ilsa{some language with pages}{some other lect also of interest} 
% There is normally no need to change the backmatter section
\backmatter
\phantomsection%this allows hyperlink in ToC to work
 \sloppy
\printbibliography[heading=bibliographie] 
% \printbibliography 
\cleardoublepage
 
\phantomsection 
\addcontentsline{toc}{chapter}{Index} 
\addcontentsline{toc}{section}{\lsNameIndexTitle}
\ohead{\lsNameIndexTitle} 
\printindex 
\cleardoublepage
  
\phantomsection 
\addcontentsline{toc}{section}{\lsLanguageIndexTitle}
\ohead{\lsLanguageIndexTitle} 
\printindex[lan] 
\cleardoublepage
  
\phantomsection 
\addcontentsline{toc}{section}{\lsSubjectIndexTitle}
\ohead{\lsSubjectIndexTitle} 
\printindex[sbj]
\ohead{} 

 
\end{document} 

% you can create your book by running
% xelatex main.tex
%
% you can also try a simple 
% make
% on the commandline

