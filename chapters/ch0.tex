\chapter{Introduction}

\section{Problématique}



Le fait que les langues sont organisées de manière économique devient évident quand on parle d’ellipse. Les utilisateurs d’une langue emploient chaque jour des structures fragmentaires ou incomplètes, en comptant sur le fait que la bonne interprétation sera de toute façon obtenue grâce au contexte ou à la connaissance du monde. Cependant, la manière dont on élide l’information considérée comme redondante n’est pas arbitraire : elle est guidée systématiquement par des facteurs syntaxiques, sémantiques et discursifs. 



Le but de cet ouvrage est essentiellement de définir les facteurs qui entrent en jeu dans la description de deux constructions elliptiques en roumain et en français. La première, appelée \textit{gapping}\footnote{{}  Le terme de \textit{gapping} est emprunté à l’anglais. Dans cet ouvrage, j’utilise généralement le terme anglais pour parler de la construction, mais quand je parle strictement de la phrase elliptique dans une construction à gapping, j’utilise le terme \textit{phrase trouée}.}, est exemplifiée en \REF{ch0:ex1a} pour le roumain et \REF{ch0:ex1b} pour le français : on coordonne une phrase complète et une phrase elliptique qui compte au moins deux éléments résiduels et où manque le verbe principal. La deuxième construction étudiée, les \textit{relatives sans verbe} (dorénavant RSV), est exemplifiée en \REF{ch0:ex2a} pour le roumain et \REF{ch0:ex2b} pour le français : à une phrase complète s’adjoint une phrase elliptique qui est introduite par un élément relatif suivi d’un ou plusieurs constituants à l’exclusion d’un verbe fini. 


\ea \label{ch0:ex1}
\ea
\gll Ioana studiază lingvistica,  {\ob}\textbf{iar}  Maria  dreptul{\cb}.    \label{ch0:ex1a}\\ 
Ioana  étudie  linguistique.\textsc{def}  et  Marie  droit\textsc{.def} \\
\glt ‘Ioana étudie la linguistique et Maria le droit.’

\ex
 Jean étudie la linguistique {\ob}\textbf{et} Marie le droit{\cb}.     \\ \label{ch0:ex1b}
\z
\z

\ea \label{ch0:ex2}
\ea
\gll Au sunat trei  persoane,  {\ob}\textbf{printre}  \textbf{care}  şi  Ion{\cb}.   \label{ch0:ex2a}\\ 
ont  appelé  trois  personnes  parmi  lesquelles  aussi  Ion\\
\glt ‘Trois personnes ont appelé, parmi lesquelles Ion.’
\ex  Plusieurs personnes, {\ob}\textbf{parmi lesquelles} Jean{\cb}, ont appelé hier.   \label{ch0:ex2b}
\z
\z



Dans les deux constructions, il s’agit d’une relation entre une séquence de constituants dont l’interprétation requiert plus que ce qui est donné par les mots qui la composent et une expression présente dans le contexte linguistique, qui fournit à cette séquence le matériel manquant dont elle a besoin pour être interprétée (en l’occurrence, le verbe \textit{étudier} dans les exemples en \REF{ch0:ex1} et le verbe \textit{appeler} dans les exemples en \REF{ch0:ex2}).



L’ellipse constitue ainsi un vrai défi à la définition \ia{Saussure, Ferdinand de}saussurienne du signe linguistique qui associe une forme (le \isi{signifiant}) à un contenu (le \isi{signifié}), car dans les constructions mentionnées ci-dessus, comme dans beaucoup d’autres, on arrive à obtenir une interprétation en l’absence d’une forme (\textit{significatio ex nihilo}). 



La question générale qui surgit alors est de savoir comment on articule cette dichotomie classique dans le cas de l’ellipse. Une solution simple, couramment admise dans les \is{grammaire générative}approches génératives, serait d’aligner ces phrases elliptiques sur leurs contreparties complètes et de considérer que le matériel qui manque a une forme «~invisible~», auquel cas l’association forme/contenu est préservée. Pour pouvoir adopter cette solution, il faut démontrer que les phrases elliptiques et leurs contreparties complètes ont exactement les mêmes propriétés. L’objectif majeur de cet ouvrage est de montrer que cette solution n’est pas adéquate pour les deux constructions en question, car le comportement de ces phrases elliptiques n’est pas toujours le même que celui des phrases complètes correspondantes. Argumenter contre l’existence d’une forme «~invisible~» dans ces structures elliptiques et, par conséquent, contre une \is{reconstruction syntaxique}reconstruction en syntaxe du matériel manquant nous demande, d’une part, de revoir l’ontologie des unités syntaxiques, ainsi que leur pertinence pour la description des phénomènes elliptiques, et, d’autre part, d’examiner l’importance d’autres facteurs. Dans cet ouvrage, j’insiste surtout sur le premier point : revoir la syntaxe et argumenter pour l’existence d’une catégorie \textit{\isi{fragment}} dans la grammaire (cf. \citealt{GinzburgEtAl2000}). La perspective syntaxique que j’adopte est donc celle résumée par \citet[5]{CulicoverEtAl2005}:

\ea 
\textit{Simpler Syntax Hypothesis}\\ 
«~The most explanatory syntactic theory is one that imputes the minimum syntactic structure necessary to mediate between phonology and meaning.~»  
\z

 
La plupart des travaux sur l’ellipse se sont concentrés sur la coordination et moins sur la subordination. Dans ce livre, j’ai choisi de traiter une construction elliptique pour chaque type de relation syntaxique, à savoir le gapping pour la coordination et les relatives sans verbe pour la subordination, pour montrer qu’une analyse uniforme (sans \isi{reconstruction syntaxique}) est disponible dans les deux types de relations syntaxiques : coordination et subordination.



Le gapping a constitué l’objet de recherches nombreuses faites sur des langues différentes (dont, en particulier, l’anglais, l’\ili{allemand}, le \ili{japonais} et le \ili{coréen}), mais il n’y a pas eu de recherche équivalente sur les \ili{langues romanes}. L’avantage de choisir le roumain comme langue privilégiée pour l’étude du gapping est double : d’une part, le roumain nous permet de confronter les \isi{contraintes de parallélisme} (tellement discutées pour le gapping) à certaines particularités typologiques (\isi{complexe verbal} riche, \isi{pro-drop}, \is{ordre de mots}ordre libre des mots, \isi{marquage casuel}, etc.) et, d’autre part, le roumain dispose d’un inventaire de conjonctions plus riche que celui des autres~langues romanes. En particulier, le roumain dispose d’une conjonction particulière \textit{iar} ‘et’, qui obéit à des contraintes spécifiques, dont certaines sont également requises de manière indépendante dans les constructions à gapping. De plus, comme cette conjonction lie uniquement des contenus propositionnels, elle est utile dans l’analyse de certaines coordinations elliptiques qui sont ambiguës entre une coordination de phrases ou bien une coordination sous-phrastique. 



En choisissant les relatives sans verbe comme deuxième construction d’étude, j’ai voulu, d’une part, faire sortir l’ellipse du domaine de la coordination (en étudiant son comportement dans la subordination) et, d’autre part, examiner une construction qui n’a pas été étudiée auparavant, mais qui est disponible dans plusieurs \ili{langues romanes}. 



\section{Contributions de cet ouvrage}
\largerpage

Sur un plan théorique, les phénomènes elliptiques ne fournissent pas tous des arguments pour supposer une structure syntaxique quelconque pour le matériel manquant. On l’admet pour certains énoncés \is{fragment}fragmentaires dans le \isi{dialogue} (cf. \citealt{GinzburgEtAl2000,GinzburgEtAl2004,Ginzburg2012}), mais cela est moins évident pour des constructions, comme les deux étudiées dans ce livre (le gapping et les relatives sans verbe), qui en apparence manifestent des effets de \is{effets de connectivité}\textit{connectivité}. Une étude approfondie nous fait découvrir que, dans les deux cas, on ne peut pas aligner ces phrases elliptiques sur leurs contreparties complètes, car (i) parfois la \isi{reconstruction syntaxique} est impossible, (ii) quand elle est disponible, elle ne peut pas s’appliquer de manière uniforme et systématique à toutes les occurrences elliptiques, et (iii) les propriétés syntaxiques et sémantiques des phrases elliptiques et de leurs contreparties complètes ne sont pas les mêmes. Comme il s’agit d’un mécanisme ad-hoc et superflu, l’hypothèse de la \isi{reconstruction syntaxique} doit être abandonnée pour ces constructions. 


  
Le refus de toute \isi{approche structurale} dans la description de ces constructions implique une perspective différente sur la syntaxe de l’ellipse. Il s’agit d’une syntaxe «~plus simple~» (dans le sens de \citealt{CulicoverEtAl2005}), sans \isi{effacement}, sans \is{élément vide}éléments vides, sans \isi{mouvement}. L’unité essentielle pour décrire le comportement de ces phrases elliptiques est le \textit{\isi{fragment}} (cf. \citealt{GinzburgEtAl2000}), défini comme une expression dont le contenu sémantique n’est pas déductible de la forme prise en isolation et dépend de l’interprétation d’un antécédent dans le contexte. Le contenu sémantique du \isi{fragment} dépend du type du \isi{fragment} (différent pour chaque construction, en raison des contraintes différentes), du contenu sémantique des constituants du \isi{fragment} et des informations contextuelles. 



La littérature sur l’ellipse est vaste et diverse aujourd’hui, mais la plupart des travaux insistent sur les points théoriques suscités par le phénomène de l’ellipse dans la grammaire et laissent en arrière-plan l’établissement et la classification des données. Dans cet ouvrage, je mets en valeur la description des données ; les points théoriques, bien que nécessaires, ne prennent pas la place de la description. L’établissement des données nous fait découvrir le rôle très important des facteurs non syntaxiques dans la description des constructions elliptiques : la sémantique (p.ex. le \is{contraste sémantique}contraste pour le gapping), le discours (p.ex. le type de \is{relation discursive}relations discursives) ou encore la \isi{structure informationnelle}.



\section{Cadre formel utilisé}



Pour la formalisation des données, j’ai choisi comme modèle théorique la grammaire syntagmatique guidée par les têtes (dorénavant HPSG, cf. anglais \textit{Head-driven Phrase Structure Grammar}). Le cadre HPSG est un formalisme grammatical qui rend compte de l’ensemble des structures bien formées d’une langue en spécifiant une hiérarchie de contraintes de bonne formation, ce qui justifie sa place parmi les \is{grammaire générative}grammaires génératives à base de contraintes. HPSG est aussi un modèle lexicaliste. Dans le sens strict du terme, cela veut dire que la syntaxe et le lexique sont séparés. Par conséquent, on ne combine pas en syntaxe les unités inférieures aux mots\footnote{{}  Cf. le principe d’intégrité lexicale \citep{MillerEtAl1997}.}  (qui sont traitées en morphologie), ce qui distingue ce modèle des \is{grammaire dérivationnelle}grammaires dérivationnelles qui manipulent les affixes dans la structure syntaxique. Dans un sens plus large, le fait d’être une \isi{grammaire lexicaliste} implique qu’au moins certaines généralisations linguistiques sont directement encodées dans le lexique. Donc, le lexique n’est pas simplement une liste d’exceptions. Ainsi, de nombreux phénomènes syntaxiques (p.ex. \isi{extraction}, phénomènes de «~montée~», alternances de valence) peuvent être décrits au moyen des \is{règle lexicale}règles lexicales, sans passer par une opération de transformation (ou dérivation). 



Le modèle HPSG est choisi dans cet ouvrage en particulier pour ses avantages liés au traitement de l’ellipse. Je reprends ici certains des avantages présentés par \citet{GinzburgEtAlToAppear}.



Comme je plaide dans cet ouvrage pour une \isi{approche non structurale} de l’el\-lipse, avec une syntaxe «~plus simple~» (sans \isi{effacement}, sans \is{élément vide}éléments vides, sans \isi{mouvement}) que ce qui est généralement postulé dans les \is{grammaire générative}grammaires géné\-ratives, le modèle HPSG convient parfaitement pour la formalisation de mes résultats, car le principe par défaut en HPSG, une conséquence du principe général d’économie (le \isi{rasoir d’Occam}, cf. \citealt{Miller1997b}), est qu’il n’y a pas de morphèmes zéro ou bien de structures syntaxiques invisibles, non prononcées. HPSG constitue un modèle surfaciste et monostratal dans lequel les expressions linguistiques sont modélisées sous forme de structures de traits typées, permettant l’organisation dans une notation commune d’informations linguistiques hétéro\-gènes ; on peut ainsi capter de façon unitaire des informations phonologiques, morphologiques, syntaxiques, sémantiques, discursives et éventuellement proso\-diques sur les mots ou les syntagmes, sans postuler d’isomorphie entre les différents niveaux d’analyse (en particulier la syntaxe et la sémantique). Une structure de traits décrit en parallèle les informations provenant de niveaux linguistiques hétérogènes, sans passer par un mécanisme de dérivation d’un niveau à l’autre. A l’aide de ces structures de traits, on représente non seulement les catégories, mais aussi les structures en constituants et les règles de grammaire. Ce type de grammaire opère avec des représentations lexicales riches et des représentations syntaxiques générales qui peuvent être sous-spécifiées. Les syntagmes sont caractérisés par la fonction grammaticale de leurs constituants immédiats (tête, sujet, complément, etc.), et non par leur catégorie ou par leur ordre.



Une propriété extrêmement puissante de ce cadre est donc le fait qu’on peut facilement modéliser simultanément des contraintes relevant de différents ni\-veaux linguistiques (p.ex. les contraintes sémantiques et syntaxiques liées à l’el\-lipse). En plus, il fournit des moyens pour intégrer des informations contextuelles. De cette manière, le cadre HPSG capte à la fois la complexité linguistique et la complexité contextuelle.



De manière générale, dans le cadre HPSG on considère qu’il n’y a pas un seul mécanisme pour le traitement de l’ellipse, mais plutôt une variété de constructions, chaque construction elliptique ayant ses conditions spécifiques de légitimation. Dans ce sens, le modèle HPSG, dans ses versions récentes \is{approche constructionnelle}constructionnelles (cf. \citealt{Sag1997,GinzburgEtAl2000,SagEtAl2003,Sag2012}), nous permet de représenter simultanément~les propriétés générales communes à une famille de constructions. Il possède une hiérarchie de constructions, qui permet de représenter non seulement les propriétés communes, mais aussi les éventu\-elles propriétés idiosyncratiques (généralement non compositionnelles) des expressions. 



Pour faciliter la lecture, dans ce livre, je distingue autant que possible la partie descriptive de la partie analytique. C’est pour cela que la formalisation en HPSG apparaît généralement en fin de chapitre. 



\section{Plan du livre}



Cet ouvrage est composée de trois grands chapitres.  \newline

\textbf{Chapitre 2. Les phrases elliptiques} 

Ce chapitre se veut une présentation générale du phénomène de l’ellipse, tel qu’il est étudié dans la littérature. Après avoir délimité ce qu’est l’ellipse, je montre qu’elle ne peut pas toujours être expliquée, comme cela a été proposé, par le principe du «~moindre effort~». Il y a bien des contextes dans lesquels l’ellipse est obligatoire ou encore des contextes dans lesquels la présence de l’ellipse entraîne des propriétés différentes qu’on ne retrouve pas dans leurs contreparties complètes. 



Comme les deux constructions elliptiques étudiées mettent en jeu respectivement une coordination et une subordination de phrases, il résulte que l’unité syntaxique \textit{phrase} est au cœur de cet ouvrage. Je lui dédie donc une partie de ce chapitre, afin de proposer une définition satisfaisante, qui nous permet de délimiter facilement les phrases elliptiques des \is{phrase averbale}phrases averbales. Je montre en particulier que la notion de \isi{tête prédicative} ne doit pas être nécessairement corrélée à la notion de phrase verbale finie ; elle est pertinente aussi dans les phrases verbales non finies ou encore dans les \is{phrase averbale}phrases averbales. 



Ensuite, je dresse une typologie des phrases elliptiques, en les classifiant par rapport à la nature du matériel manquant, le type de contexte syntaxique dans lequel l’ellipse apparaît, ainsi que par rapport à la directionnalité de l’ellipse. Je donne en particulier un aperçu des différentes constructions elliptiques disponi\-bles dans les deux langues romanes étudiées dans cet ouvrage (le roumain et le français). Je montre ensuite la difficulté d’une étude sur l’ellipse dans une perspective typologique, quant à la disponibilité des différentes constructions inventoriées et aux problèmes d’identification de certaines constructions. Le reste du chapitre présente le grand débat de la littérature entre les différentes manières d’envisager la résolution de l’ellipse, réduites ici à la \isi{reconstruction syntaxique} (dans les \is{approche structurale}approches structurales) vs. \isi{reconstruction sémantique} (dans les approches \is{approche non structurale}non structurales). \newline


\textbf{Chapitre 3. Les conjoints fragmentaires : le gapping} 


Le troisième chapitre est dédié aux coordinations à gapping en roumain et en français. Je montre que la définition classique du gapping, qui retient parmi ses critères la position médiane du verbe manquant et la présence d’un élément résiduel sujet, n’est pas adéquate. Adopter une définition plus large en termes de couverture empirique complique en revanche le travail de délimitation du gapping (à l’intérieur d’une même langue et à travers les langues). D’une part, en roumain et en français, le gapping pose des problèmes de description par rapport à d’autres constructions elliptiques, en particulier les \isi{comparatives} elliptiques et le \is{Stripping}stripping. D’autre part, en roumain, les configurations avec le verbe prédicat en position initiale se prêtent a priori à une double analyse : gapping (donc, une coordination au niveau des phrases) ou bien coordination de séquences (et donc une coordination sous-phrastique dans la portée syntaxique d’un prédicat). 



Dans un premier temps, je laisse de côté les configurations ambiguës et je présente les propriétés du gapping dans les contextes non ambigus ({\cad} avec un verbe prédicat en position médiane ou finale). J’observe d’abord les contraintes générales qui s’appliquent au matériel manquant (en particulier, le type d’identité qui s’établit entre le verbe antécédent et le verbe manquant) et ensuite les contraintes auxquelles obéissent les éléments résiduels. Les \isi{contraintes de parallélisme}, souvent discutées dans la littérature, sont reprises ici en détail. Je montre que le parallélisme le plus strict opère au niveau sémantico-discursif. Sur le plan syntaxique, différentes \is{asymétrie syntaxique}asymétries peuvent apparaître, à condition que chaque conjoint puisse apparaître seul en lieu et place de la coordination dans son ensemble (selon la \is{généralisation de Wasow}généralisation dite «~de \ia{Wasow, Thomas}Wasow~», cf. \citealt{GazdarEtAl1985} et \citealt{PullumEtAl1986}). L’importance des différentes contraintes pesant sur le gapping est évaluée pour le roumain à la lumière de la conjonction \textit{iar} ‘et’, spécialisée en roumain pour marquer le \is{contraste sémantique}contraste.



Une autre partie du chapitre est dédiée aux différentes analyses proposées dans la littérature pour le traitement du gapping, qui peuvent être regroupées en trois approches majeures : \isi{reconstruction syntaxique} (donc, une ellipse syntaxique) ; \isi{mouvement} (donc, pas d’ellipse), et \isi{reconstruction sémantique} (donc, une ellipse sémantique). Après avoir inventorié les différents arguments invoqués dans la littérature pour ou contre l’une de ces approches, je donne des arguments empiriques en faveur d’une \isi{approche constructionnelle} des coordinations à gapping ({\cad} \isi{reconstruction sémantique}) et contre les approches alternatives en termes de \isi{reconstruction syntaxique} ou \isi{mouvement}. Dans cette perspective, la construction à gapping est une coordination entre une phrase finie non elliptique et une phrase \is{fragment}fragmentaire. Une formalisation de cette approche est donnée ensuite dans le cadre HPSG. 



Dans~la dernière section du chapitre, je reviens aux configurations ambiguës en roumain, dans lesquelles le verbe de la phrase complète se trouve en position initiale. Je montre que le problème de l’ambiguïté est résolu pour les coordinations en \textit{iar} ‘et’, ce type de structure recevant la même analyse que les distributions typiques de gapping. En revanche, pour les coordinations avec d’autres conjonctions, l’analyse est systématiquement ambiguë entre une coordination phrastique (avec ellipse) et une coordination sous-phrastique (sans ellipse). \newline 


\textbf{Chapitre 4. Les subordonnées fragmentaires : les relatives sans verbe}



Ce dernier chapitre est consacré à l’étude des relatives sans verbe (RSV) en roumain et en français. Dans un premier temps, je discute leurs propriétés syntaxiques, en regardant la constituance (en particulier, les propriétés distributionnelles de l’introducteur et celles du corps d’une RSV) et la \isi{linéarisation}, pour en conclure que les RSV se comportent, dans les deux langues, comme des \is{ajout incident}ajouts incidents par rapport à la phrase hôte. Dans un deuxième temps, je m’intéresse à leurs propriétés sémantiques. Je montre que les RSV ont un comportement hybride, car elles ont une interprétation non intersective (comme les relatives non restrictives), mais leur contenu fait partie du contenu asserté (comme dans le cas d’une relative restrictive). De plus, leur sémantique est partitive, rendant possible deux types d’interprétations : une \isi{interprétation exemplifiante} et une \isi{interprétation partitionnante}. Les deux interprétations me permettent de décrire plus précisement les différences qu’on observe d’un introducteur à l’autre en roumain et en français, ainsi que les préférences des locuteurs pour certains introducteurs dans certaines configurations des RSV.


Après avoir observé les propriétés syntaxiques et sémantiques des RSV, je montre que l’hypothèse selon laquelle les RSV sont dérivées à partir des phrases relatives ordinaires est empiriquement intenable. D’une part, la \isi{reconstruction syntaxique} d’une forme verbale n’est pas toujours possible et, quand elle est disponible, des contraintes lexicales, syntaxiques ou sémantiques doivent être prises en compte au cas par cas. Donc il n’y a pas de mécanisme général de \isi{reconstruction syntaxique} qui s’applique à toutes les RSV. D’autre part, la contribution sémantique des RSV n’est pas la même que celle de leurs contreparties complètes. Par conséquent, je considère que les RSV ne mettent pas en jeu une ellipse syntaxique et que toutes les différences enregistrées entre les phrases relatives verbales et les RSV s’expliquent par le fait que les RSV sont des ajouts fragmentaires. A la fin du chapitre, je montre comment une \isi{approche constructionnelle} des RSV en termes de \textit{\is{fragment}fragments} peut être formalisée dans le cadre HPSG, en utilisant le langage \is{Minimal Recursion Semantics (MRS)}MRS (angl. \textit{Minimal Recursion Semantics}), qui permet, entre autres, la description des représentations sémantiques incomplètes. 

 