\chapter{Conclusion générale} \label{concl}


Dans cet ouvrage, j’ai développé un fragment de grammaire qui rend compte des propriétés majeures de deux constructions elliptiques (auxquelles «~manque~» la tête verbale), appelées respectivement gapping \REF{concl:ex1a} et relatives sans verbe \REF{concl:ex1b}, abrégées RSV. Les deux constructions ont été choisies pour montrer qu’une analyse uniforme (sans \isi{reconstruction syntaxique}) est disponible dans les deux types de relations syntaxiques : coordination et subordination. Au terme de ce livre, je veux insister sur les aspects généraux engendrés par l’étude des phrases elliptiques. 

\ea \label{concl:ex1}
\ea
Jean aime les pommes [\textbf{et} Marie les bananes]. \label{concl:ex1a}
\ex
Plusieurs personnes sont venues cette semaine, [\textbf{dont} Marie (hier)]. \label{concl:ex1b}
\z 
\z 


\section{Ellipse et reconstruction} \label{concl:sect1}


La proposition centrale défendue ici est que les phrases trouées dans les coordinations à gapping et les subordonnées relatives sans verbe ne peuvent pas être alignées sur le fonctionnement d’une phrase verbale ordinaire. Leurs propriétés syntaxiques et sémantiques montrent qu’elles ne sont pas dérivées à partir d’une phrase complète. Une analyse en termes de \isi{reconstruction syntaxique} est donc inadéquate. Par conséquent, la phrase elliptique dans les deux types de structures mentionnés comporte un mode d’organisation syntaxique spécifique et doit avoir un statut indépendant dans la grammaire, à savoir le statut d’une \is{fragment}\textit{phrase fragmentaire}, {\cad} une unité syntaxique qui a un contenu de type message, mais dont la syntaxe est incomplète. Cet ouvrage apporte de nouveaux arguments en faveur d’une \is{reconstruction sémantique}reconstruction plutôt sémantique avec des \isi{contraintes de parallélisme}, cf. \citet{GinzburgEtAl2000} et \citet{CulicoverEtAl2005}.

Fondamentalement, la sémantique complète peut être obtenue à partir d’une syntaxe incomplète, en exploitant la notion de \is{fragment}\textit{fragment}, comme l’avaient proposé \citet{GinzburgEtAl2000} pour les \is{question courte}questions et les \is{réponse courte}réponses courtes dans le \isi{dialogue} en anglais. On se donne donc en syntaxe la notion de \is{fragment}\textit{fragment} conçu comme une construction à laquelle sont associées des conditions de bonne formation syntaxiques et interprétatives. Cependant, contrairement à \citet{GinzburgEtAl2000} qui analysent les \is{fragment}fragments phrastiques comme ayant l’ensemble des propriétés d’une phrase finie (cf. la catégorie VERBAL), j’ai choisi de représenter les \is{fragment}fragments comme étant construits à partir d’un \is{cluster}\textit{cluster}, notion reprise de \citet{Mouret2006,Mouret2007} et requise de manière indépendante pour les coordinations de séquences dans la portée syntaxique d’un prédicat. Le syntagme de type \isi{cluster} réunit tous les éléments résiduels d’une phrase elliptique et rend accessibles leurs propriétés syntaxiques et sémantiques au niveau de la construction. En permettant au \isi{cluster} de comporter un seul constituant immédiat ou plus, on peut obtenir une analyse uniforme des coordinations à gapping et des RSV dont le corps est composé d’un ou plusieurs constituants immédiats. Le \isi{fragment} hérite la catégorie sous-spécifiée du \isi{cluster}, ce qui lui permet de se combiner avec des foncteurs sélectionnant des catégories non finies (p.ex. \textit{ainsi que} et \textit{non pas} en français)\footnote{ Ces deux notions sont compatibles avec plusieurs traitements, comme on a pu le voir dans le chapitre~\ref{ch2} et chapitre~\ref{ch3}~: dans le chapitre~\ref{ch2}, j’ai tout simplement utilisé une version \is{approche constructionnelle}constructionnelle de HPSG, alors que dans le chapitre~\ref{ch3}, j’ai utilisé, en plus, le langage \is{Minimal Recursion Semantics (MRS)}\textit{Minimal Recursion Semantics}.} . 

Une fois les deux notions de \is{fragment}\textit{fragment} et \is{cluster}\textit{cluster} introduites dans la grammaire, on peut ajouter les contraintes spécifiques à chaque construction. Ainsi, une coordination à gapping se distingue des autres constructions elliptiques par le fait que la phrase trouée doit suivre la phrase source (en roumain et en français)~et par le fait que la \isi{relation discursive} qui s’établit entre les conjoints est toujours une relation symétrique. Quant aux propriétés spécifiques des RSV, on note la présence d’un introducteur avec une sémantique partitive ({\cad} il impose une relation de type ensemble/sous-partie).

Fondamentalement, la description et l’analyse des deux constructions étudiées dans ce livre montrent que, de manière générale, on peut envisager une grammaire syntagmatique simple, surfaciste, sans \is{élément vide}éléments vides ou \isi{effacement}, sans \isi{mouvement} et sans postuler nécessairement d’homomorphisme syntaxe-séman\-tique.

\section{Ellipse et parallélisme}\label{concl:sect2}

Un des arguments majeurs qu’on mentionne habituellement en faveur d’une \isi{reconstruction syntaxique} dans les constructions elliptiques est la présence des \is{effets de connectivité}effets de «~connectivité~» discutés dans la section~\ref{ch1:sect1.5.1.1}, {\cad} un parallélisme structural entre la phrase elliptique et la phrase source, en ce qui concerne les propriétés morpho-syntaxiques des éléments résiduels. 

\newpage 
Les résultats de cette recherche montrent que le parallélisme structural, tel qu’il est discuté dans les travaux sur l’ellipse, est moins strict que ce que l’on pense (\textit{contra} \citealt{CulicoverEtAl2005}). Ainsi, pour les constructions à gapping, on a vu que le \is{contraintes de parallélisme}parallélisme syntaxique n’est pas strict en ce qui concerne la catégorie grammaticale, le nombre de dépendants réalisés, ainsi que \is{ordre de mots}l’ordre dans lequel apparaissent les éléments résiduels par rapport à leurs corrélats dans la phrase source. Ce \is{contraintes de parallélisme}parallélisme syntaxique «~relâché~» exige simplement que les éléments résiduels remplissent les conditions de sélection du prédicat antécédent dans la phrase source (cf. la \is{généralisation de Wasow}généralisation de \ia{Wasow, Thomas}Wasow qui gère les \is{coordination de termes dissemblables}coordinations de termes dissemblables). De même, dans les RSV, on observe des \is{asymétrie syntaxique}asymétries en ce qui concerne le \isi{marquage prépositionnel} (et \is{marquage casuel}casuel en roumain) : ainsi, en français, de manière générale, l’élément distingué dans le corps de la RSV ne peut pas recevoir le \isi{marquage prépositionnel} de son légitimeur dans la phrase hôte ; en roumain aussi, l’élément distingué dans les RSV introduites par \textit{dintre care} ‘parmi lesquel(le)s’ ne reçoit généralement pas de marque casuelle ou prépositionnelle, par rapport à son légitimeur.

En revanche, on doit accorder plus d’attention aux \is{contraintes de parallélisme}effets de parallélisme au niveau sémantique (et discursif pour le gapping). De manière générale, ces cons\-tructions elliptiques mettent en jeu un parallélisme sémantique fort. Dans les constructions à gapping, il doit y avoir au moins deux \is{contraste sémantique}contrastes sémantiques entre les éléments résiduels et les corrélats ({\cad} deux \is{paire contrastive}paires contrastives). De plus, pour les coordinations à gapping, un parallélisme fort est observé aussi au niveau discursif, le gapping privilégiant les \is{relation discursive}relations symétriques de parallélisme et de contraste. Dans les RSV, la relation sémantique qui s’établit entre l’élément distingué dans le corps de la RSV et son légitimeur dans la phrase hôte est toujours une relation partitive ({\cad} le légitimeur dans la phrase hôte doit être une entité fractionable exprimant une somme dont les sous-parties sont accessibles, alors que l’élément distingué dans la RSV doit être interprété comme une sous-partie de l’entité fractionable). 


\section{Perspectives}\label{concl:sect3}


Un premier point que je veux mentionner concerne la description du gapping. Afin d’avoir une analyse complète des coordinations à gapping, le travail présenté dans ce livre devrait être suivi par une étude prosodique. On a vu que le \is{contraintes de parallélisme}parallélisme est strict surtout au niveau sémantique et discursif. Il reste à vérifier si l’intonation dans le gapping est plutôt sensible aux aspects sémantiques et discursifs et moins aux aspects syntaxiques, ce qui invaliderait l’hypothèse de \citet{FeryEtAl2005}. On devrait vérifier aussi de plus près, dans les deux langues, les différents facteurs qui influencent les préférences des locuteurs dans l’acceptabilité des exemples, et cela peut être mieux observé en faisant des études de corpus et en utilisant des méthodes expérimentales.

Une deuxième piste de recherche concerne la parenté qui pourrait être établie entre les deux constructions étudiées dans cet ouvrage et d’autres constructions elliptiques. A plusieurs reprises, j’ai émis l’hypothèse selon laquelle certaines constructions elliptiques semblent se prêter à un même type d’analyse que celle proposée pour le gapping et les RSV. 

Un type d’ellipse proche du gapping est le \is{Stripping}stripping et en particulier les ellipses polaires, dans lesquelles l’élément résiduel est accompagné d’un adverbe polaire comme \textit{aussi} \REF{concl:ex2a} et \textit{non plus} \REF{concl:ex2b} en français. Comme je l’ai déjà précisé dans le chapitre~\ref{ch2}, certains auteurs (\citealt{HankamerEtAl1976,Gardent1991,Lobeck1995,Hartmann2000,Toosarvandani2011}, etc.) analysent ces exemples comme un sous-type de gapping. Une description détaillée doit être faite pour voir si l’on peut trouver une analyse uniforme pour le gapping et les ellipses polaires.

\ea \label{concl:ex2}
\ea 
Jean viendra à la fête [\textbf{et} Marie \uline{aussi}]. \label{concl:ex2a} 
\ex
Jean n’est pas venu à la fête [\textbf{et} Marie \uline{non plus}]. \label{concl:ex2b}
\z 
\z

Les structures \isi{comparatives} \REF{concl:ex3} constituent un autre type d’ellipse qui permet des séquences qui ressemblent au gapping dans la coordination (cf. \citealt{Zribi-Hertz1986,CulicoverEtAl2005,AmsiliEtAl2008}). Toujours dans le chapitre~\ref{ch2}, j’ai mentionné quelques éléments suggèrant la souplesse des contraintes sur les structures \isi{comparatives}, par rapport à celles qui sont en jeu dans une structure coordonnée. Une étude détaillée reste à faire pour voir si l’on peut envisager une analyse uniforme.

\ea \label{concl:ex3}
\ea 
Jean est autant doué en bricolage [\textbf{que} Marie en décoration]. 
\ex
Il s’ennuie chez lui, [\textbf{comme} moi au boulot]. \citep{AmsiliEtAl2008}
\z 
\z 

Enfin, en dehors des RSV étudiées dans le chapitre~\ref{ch3}, il y a d’autres subordonnées elliptiques ayant la fonction d’ajout et permettant des séquences à deux éléments résiduels : les ajouts additifs \REF{concl:ex4a}, les ajouts exceptifs \REF{concl:ex4b} ou encore les ajouts concessifs \REF{concl:ex4c}. Là encore, une étude reste à faire pour chaque construction, pour voir si l’analyse proposée dans ce livre peut être étendue.

\ea \label{concl:ex4}
\ea 
Tout le monde a apporté quelque chose, [\textbf{y compris} Marie un gâteau]. \label{concl:ex4a} 
\ex 
Personne n’a apporté quoi que ce soit, [\textbf{sauf} Marie un gâteau]. \label{concl:ex4b}
\ex {}
[\textbf{Bien que} pour la première fois à l’étranger], Marie s’est très bien débrouillée. \label{concl:ex4c}
\z 
\z

Contrairement à ce que l’on peut penser, j’aimerais préciser que les résultats de cette recherche ne remettent pas en cause à eux seuls la nécessité d’un mécanisme de \isi{reconstruction syntaxique} dans la grammaire. Les mises en facteur à droite (abrégées \is{Right-Node Raising (RNR)}RNR), par exemple, semblent mettre en jeu un mécanisme d’ellipse syntaxique \citep{AbeilleEtAl2010,Chaves2014,AbeilleEtAl2016}. Comme je l’ai déjà mentionné dans la conclusion du chapitre~\ref{ch1}, je considère que dans une grammaire de l’ellipse les deux solutions, à savoir la \isi{reconstruction syntaxique} et la \is{reconstruction sémantique}reconstruction à l’interface syntaxe-sémantique, doivent être disponibles, afin de rendre compte des propriétés des différentes constructions elliptiques.