\title{Grammaire des constructions elliptiques}  %look no further, you can change those things right here.
\subtitle{Une étude comparative des phrases sans verbe en roumain et en français} 
\BackBody{Dans cet ouvrage, j’ai développé un fragment de grammaire qui rend compte
des propriétés majeures de deux constructions elliptiques (auxquelles
«~manque~» la tête verbale), appelées respectivement gapping
(1a) et relatives sans verbe (1b).
Les deux constructions ont été choisies pour montrer qu’une analyse
uniforme (sans reconstruction syntaxique) est disponible dans les deux
types de relations syntaxiques : coordination et subordination.

\ea
\ea
Jean aime les pommes [\textbf{et} Marie les bananes].  
\ex
Plusieurs personnes sont venues cette semaine, [\textbf{dont} Marie
(hier)].  
\z
\z

La proposition centrale défendue ici est que les phrases elliptiques dans
les coordinations à gapping et les subordonnées relatives sans verbe ne
peuvent pas être alignées sur le fonctionnement d’une phrase verbale
ordinaire. Leurs propriétés syntaxiques et sémantiques montrent qu’elles ne
sont pas dérivées à partir d’une phrase complète. Une analyse en termes de
reconstruction syntaxique est donc inadéquate. Par conséquent, la phrase
elliptique dans les deux types de structures mentionnés comporte un mode
d’organisation syntaxique spécifique et doit avoir un statut indépendant
dans la grammaire, à savoir le statut d’une \textit{phrase fragmentaire},
c.-à-d. une unité syntaxique qui a un contenu de type \textit{message}, mais dont la
syntaxe est incomplète. Cet ouvrage apporte de nouveaux arguments en faveur
d’une reconstruction plutôt sémantique avec des contraintes de
parallélisme, cf. \citet{GinzburgEtAl2000} et \citet{CulicoverEtAl2005}.}
%\dedication{Change dedication in localmetadata.tex}
\proofreader{Margot Colinet, Frédéric Laurens,
Sebastian Nordhoff, Florence Orluc, Delphine Tribout, Esma Tanis, Grégoire Winterstein}
\typesetter{Gabriela Bîlbîie, Sebastian Nordhoff, Felix Kopecky, Iana Stefanova}
\author{Gabriela Bîlbîie} 
\renewcommand{\lsISBNdigital}{978-3-944675-52-7}
\renewcommand{\lsISBNhardcover}{978-3-946234-82-1}
\renewcommand{\lsISBNsoftcover}{978-3-944675-67-1} 
\renewcommand{\lsSeries}{eotms} % use lowercase acronym, e.g. sidl, eotms, tgdi
\renewcommand{\lsSeriesNumber}{2} %will be assigned when the book enters the proofreading stage
\renewcommand{\lsURL}{http://langsci-press.org/catalog/book/27} % contact the coordinator for the right number
\newcommand{\lsID}{27}
\BookDOI{10.5281/zenodo.573753}

 